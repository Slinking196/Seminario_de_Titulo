\chapter{Marco Teórico}
\label{cha:marco_teorico}

En esta sección, se explorarán los conceptos y teorías relevantes que sustentan la investigación. Se abordarán temas como la experiencia del usuario (UX), la experiencia del consumidor (CX) y la experiencia del turista (TX), así como sus interrelaciones y dimensiones.

\section{Usuario}
\label{sec:usuario}

El usuario es cualquier persona que interactúa directa o indirectamente con un producto interactivo, ya sea para alcanzar un objetivo personal o profesional [\cite{preece2015interaction}]. En el contexto de esta investigación, el usuario es el turista que utiliza plataformas digitales para planificar y llevar a cabo su viaje. Comprender al usuario implica analizar sus necesidades, expectativas y comportamientos a lo largo de su experiencia turística.

\section{Consumidor}
\label{sec:consumidor}
Definido en \cite{becker2020customer} en un contexto relacionado a CX, Customer o cosumidor es considerado "como la persona que recorre el customer journey, es decir, quien está expuesto a los estímulos de la empresa, marca, producto o servicio" en cualquier momento, ya sea antes durante o despues de la exposición al estimulo pero dentro del customer journey.

Aunque acercandonos un poco mas a la definición asociada a un turista como extensión o derivado de consumidor, este último  

\section{Customer Journey}
Como define la norma \cite{ISO24082} corresponde a la \textit{serie o conjunto de experiencias del cliente al interactuar con una organización, sus productos o servicios}, en el fondo toda la experiencia percibida por el consumidor al momento de la interacción.




\section{Turismo}
\label{sec:turismo}
Definído por \cite{unwto-glossary} (aunque tambien llamados visitantes como sinónimo) El turismo es considerado como un \textit{fenómeno social, cultural y económico que conlleva inevitablemente el movimiento de personas a países o lugares fuera de su entorno habitual por razones personales o de negocios/profesionales.} Por lo que podemos considerarlo como el conjunto de actividades realizadas por personas que viajan y permanecen temporalmente fuera de su entorno habitual por motivos de ocio, negocios u otros propósitos.

\section{Turista}
\label{sec:turista}


Llamamos \emph{turista} al visitante y cliente principal del turismo, que se desplaza fuera de su entorno habitual y pasa \emph{al menos una noche} en el destino\parencite{ortiz2024slr}; su permanencia es temporal (menos de un año) y no conlleva actividad laboral remunerada en el lugar visitado .

\section{Experiencia del Usuario (UX)}
\label{sec:ux}

La La \emph{experiencia de usuario (UX)} se define como las percepciones y respuestas de una persona que resultan del uso \emph{y/o} del uso anticipado de un sistema, producto o servicio; el concepto amplía la usabilidad al incorporar emociones, expectativas y contexto \parencite{ISO9241-210:2010}.

\section{Experiencia del Consumidor (CX)}
\label{sec:cx}

La experiencia del consumidor (CX) es un concepto más amplio que incluye todas las interacciones que un cliente tiene con una marca a lo largo de su relación. Esto abarca no solo la interacción con productos y servicios, sino también la comunicación y el soporte al cliente. La CX se considera una extensión de la UX, ya que ambas se centran en la satisfacción y percepción del usuario.


CX, puede ser considerado por muchos autores como Laming \& Mason como una extensión de UX, altamente multidisciplinaria, que considera las respuestas de los usuarios tanto antes como durante y después de la interacción con una marca/compañía y durante toda una jornada [\cite{laming2014customer_experience}]. También mencionan cómo la CX debe ser definida como las experiencias tanto físicas como emocionales generadas antes, durante o después, pero producto de la interacción con un producto o servicio de una marca. Además, aparecen conceptos como el objetivo de las marcas para mejorar la CX o que el indicador clave es la satisfacción del cliente (Customer Satisfaction).

\section{Experiencia del Turista (TX)}
\label{sec:tx}

La experiencia del turista (TX) es un caso particular de la CX, que se centra en las interacciones y percepciones de los turistas antes, durante y después de su viaje. La TX abarca múltiples dimensiones, incluyendo emociones, sensaciones y experiencias relacionadas con el viaje. Este concepto es especialmente relevante en el contexto de la industria del turismo, donde la satisfacción del cliente puede verse afectada por diversos factores.
Según Matus et al. (2023), la TX puede definirse como "las percepciones subjetivas cognitivas, afectivas, sensoriales y conativas, ya sean negativas o positivas, e influenciadas por factores situacionales, que tiene un turista al interactuar con las marcas antes, durante y después del viaje, incluyendo sus resultados" (p. 10).

Sus dimensiones son:
\begin{itemize}
  \item \textbf{Hedonismo}: El componente placentero de la vivencia turística—diversión, disfrute y sentimientos positivos que hacen “grato” el episodio.
  \item \textbf{Renovación (refreshment)}: Sensación de descanso, alivio y recarga mental/física; el viaje como “respiro” que restaura.
  \item \textbf{Cultura local}: Contacto significativo con costumbres, gastronomía, modos de vida y patrimonio del destino; cuanto más cercano y auténtico, más memorable.
  \item \textbf{Significado (meaningfulness)}: Relevancia personal de lo vivido—reflexión, propósito y lo que la experiencia “deja” en valores o identidad.
  \item \textbf{Conocimiento (knowledge)}: Aprendizaje adquirido sobre el lugar, su historia/cultura o sobre uno mismo; incorporación de nueva comprensión.
  \item \textbf{Involucramiento (involvement)}: Grado de participación y compromiso del turista con actividades y personas; atención, implicación y “estar metido” en la experiencia.
  \item \textbf{Novedad (novelty)}: Percepción de descubrimiento y ruptura de la rutina; elementos inusuales o únicos que diferencian el viaje de lo cotidiano.
\end{itemize}

\textit{Nota metodológica.} Estas \emph{siete} dimensiones se validan como un \emph{modelo de primer orden}; no conviene colapsarlas en macro-factores (p. ej., “afectivo/cognitivo/conductual”) al evaluar, porque la estructura de segundo orden no mejora el ajuste del modelo.
\parencite{kim2012development}

\section{Procesamiento de lenguaje natural (NLP)} 
\label{sec:nlp}

Definida por \cite{Khurana_2022},  Procesamiento de lenguaje natural (NLP, por sus siglas en inglés)\textit{''Es un campo interdisciplinario que se enfoca en la interacción entre computadoras y el lenguaje humano, con el objetivo de permitir que las máquinas comprendan, interpreten y generen texto o habla de manera significativa.''}. Corresponde a una rama de la inteligencia artificial que se cruza con la lingüística y que busca realizar análisis de las palabras en oraciones gracias al Machine Learning(ML) y Deep Learning(DL), necesitando preprocesamiento de datos para que funcione correctamente, con técnicas como tokenización, lematización y eliminación de stopwords.