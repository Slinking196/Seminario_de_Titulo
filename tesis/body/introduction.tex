\chapter{Introducción}
\label{cha:introduccion}

En el último siglo las tecnologias han avanzado a pasos agigantados, las nuevas tecnologias, dispositivos, el uso de IoT y Edge Computing, la IA generativa últimamente que ha generado cambios completos en la sociedad y que da para un survey completo de estado del arte, Paradigmas o arquitecturas como Zero Trust entre muchas otras tecnologias han cambiado la forma en que la sociedad vive, convive, se expresa, siente o incluso viaja. Es aquí donde debemos hablar sobre como las personas (a resumidas cuentas) perciben y sienten los touchpoints de las aplicaciones y/o sistemas, que aunque podria tener una definición mas amplia es donde podemos reconocer el concepto de UX.\\

El concepto de UX ha sido 





El concepto de UX puede abarcar muchos puntos, pudiendo ser definida abarcando distintos aspectos de la experiencia de los usuarios, pero a modo general podemos definirla como un conjunto de percepciones, emociones, respuestas, tanto físicas como psicológicas de los usuarios antes, durante y después del uso de una aplicación (aunque principalmente deberiamos concentrarnos en el durante representado por los touchpoints de los usuarios con el sistema, ya que representan el momento en que comúnmente se asocia una buena o mala UX del sistema). Considera todo tipo de interacción del usuario (sobre todo touchpoints) con el sistema, factores como la usabilidad son muy importantes, accesibilidad, satisfacción general, hedonismo tambien van relacionados y podriamos considerar muchos factores mas, ya que la UX y sus derivados son conceptos multidimensionales y que necesita un acercamiento mas holístico para llegar a un marco mas ampio, sobre todo porque sus derivados son iguales o se ven influenciados por muchos factores como lo son la Customer Experience (CX por sus siglas en Ingles)\\  


CX, pudiendo ser considerado por muchos autores como como una extensión de UX

TX es una disciplina que corresponde a un caso particular de CX (A su vez relacionada con UX), 
pero con la particularidad de que estudia a los sentimientos, sensaciones, experiencias y muchas mas dimensiones 
asociadas a la relacion Turista - Viaje en los momentos antes, durante y despues del mismo, relacion 
que puede verse a su vez afectada por multiples factores, otorgandole una dimensionalidad infinita, 
considerando que en el ámbito de los viajes el turista puede verse afectado positiva o negativamente por tantas variables como factores influyen en nuestra vida diaria, teniendo ideas preconcebidas de los lugares que visitará incluso sin saberlo o haberse dado cuenta (sobre todo cuando se trata de publicidad o campañas que tienenpor objetivo que sin darte cuenta implanten una idea sobre determinado tema en las personas).\\

TX posee muchas definiciones distintas y muchos autores han intentado llegar a un concenso de que es
realmente, en que consiste, que factores estudia, de donde se origina, cuantas dimensiones contempla, 
entre muchos mas factores, pero lo que si podemos decir es que podríamos considerarlo un constructo 
Holístico y multidimensional, derivado como un caso particular de CX (concepto asociado directamente a UX) donde 
el Customer utiliza servicios como productos o sistemas asociados a viajes. TX Según Matus et al. (2023) se cree que, “la Experiencia Turística (TX) puede referirse a las percepciones subjetivas cognitivas, afectivas, sensoriales y  conativas, ya sean negativas o positivas, e influenciadas por factores situacionales, que tiene un turista al  interactuar con las marcas antes, durante y después del viaje, incluyendo sus resultados.”(p. 10). por lo que podemos 
