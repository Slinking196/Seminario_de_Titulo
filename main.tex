\documentclass[12pt]{article}

% Paquetes comunes
\usepackage[utf8]{inputenc}     % Codificación UTF-8
\usepackage[T1]{fontenc}        % Codificación de fuentes
\usepackage[spanish]{babel}     % Idioma español
\usepackage{amsmath, amssymb}   % Matemáticas avanzadas
\usepackage{graphicx}           % Imágenes
\usepackage{hyperref}           % Enlaces

\title{Mi Primer Documento en \LaTeX}
\author{Tu Nombre}
\date{\today}

\begin{document}

\maketitle

\section{Introducción}

Este es un documento de ejemplo creado con \LaTeX{} en macOS usando la terminal. Puedes compilarlo con:

\begin{verbatim}
pdflatex ejemplo.tex
\end{verbatim}

\section{Fórmulas matemáticas}

Aquí tienes una fórmula famosa:

\[
e^{i\pi} + 1 = 0
\]

También puedes incluir ecuaciones numeradas:

\begin{equation}
a^2 + b^2 = c^2
\end{equation}

\section{Imágenes}

Puedes insertar imágenes si tienes un archivo como `imagen.jpg` en el mismo directorio:

\begin{figure}[h]
  \centering
  \includegraphics[width=0.4\textwidth]{imagen.jpg}
  \caption{Una imagen de ejemplo.}
\end{figure}

\section{Enlaces}

Aquí un enlace a \href{https://www.latex-project.org/}{LaTeX Project}.

\end{document}
