\chapter{Enfoque de la investigación}
\label{cha:enfoque_investigacion}

En esta sección definiremos el enfoque de la investigación, considerando el tipo de estudio, los datos que tenemos, las técnicas de análisis de datos que utilizaremos, la forma en que llegaremos a las conclusiones, entre otros aspectos que nos permitan definir claramente un enfoque que puede ser tanto cuantitativo como cualitativo o mixto.\\


Sampieri en su obra \textit{Metodología de la investigación, 6ta edición}\cite{sampieri2014metodología}, describe que el enfoque cuantitativo se caracteriza en que \textit{utiliza la recolección de datos para probar hipótesis con base en la medición numérica y el análisis estadístico, con el fin establecer pautas de comportamiento y probar teorías}, adecuandose muy bien al análisis de datos numéricos, empíricos, visibles a simple vista y de naturaleza objetiva.\\

Además, menciona respecto al enfoque cualitativo que este \textit{utiliza la recolección y  análisis de los datos para afinar las preguntas de investigación o revelar nuevas interrogantes en el proceso de interpretación}, considerando los aspectos que no pueden ser analizados con simples métricas, fórmulas o comparaciónes cuantitativas u objetivas, como lo es el caso del análisis y comprensión de texto con NLU, la detección de lo que consideramos como "Sugerencia" u otros aspéctos de naturaleza subjetiva y que no se reconozcan o analicen de manera directa o a simple vista.\\

Sampieri además de definir, menciona la razón por la que se deben utilizar enfoques mixtos (Enfoque que combina los dos anteriores mencionados), ya que los problemas o fenómenos científicos (extrapolando a nuestro caso que es el área de TX) \textit{representan o están constituidos por dos realidades, una objetiva y la otra subjetiva} y que \textit{para poder “capturar” ambas realidades coexistentes (la realidad intersubjetiva), se requieren tanto la visión “objetiva” como la “subjetiva”}\cite{sampieri2014metodología}.

Con esto claro, podemos afirmar entonces que en esta investigación nos concentraremos en un trabajo y análisis de datos cuantitativo y cualitativo, teniendo en cuenta 2 argumentos principales: Por un lado el análisis y comprensión de texto (NLU) de comentarios de personas (análisis puramente cualitativo) y por el otro el análisis estadístico de los datos obtenidos luego del trabajo de datos con modedlos NLU, que presentarán métricas, valores, errores y datos estadísticos puramente cuantitativos, además de los propios datos iniciales del dataset como el rating de cada comentario/review (análisis cuantitativo), por lo que podemos deducir que el enfoque debería ser mixto, considerando análisis tanto cualitativo como cuantitativo de manera complementaria.\\

Cabe destacar que el enfoque mixto puede subdividirse en distintos grupos de estudio mixtos como menciona Sampieri \cite{sampieri2014metodología} en la figura \ref{fig:3 tipos de enfoques}, dependiendo de cual de los dos métodos (Cualitativo o Cuantitativo) predomine en el estudio, y en nuestro caso, el análisis cuantitativo podría llegar a predominar por sobre el cualitativo, debido a que el análisis de datos cuantitativo que obtendrémos luego de aplicar los modelos NLU son los datos que nos otorgarán gráficos, tablas comparativas y conclusiones más sólidas y verdaderamente útiles, además de que este abarcará un mayor tiempo del trabajo de investigación, podemos considerar el método más específicamente como CUAN-cual.\\

\begin{figure}[H]
    \centering
    \includegraphics[width=\textwidth]{assets/3-métodos-principales.png}
    \caption{ tres principales enfoques de la investigación hoy en día, incluyendo subtipos de estudios mixtos.}
    \label{fig:3 tipos de enfoques}
\end{figure}