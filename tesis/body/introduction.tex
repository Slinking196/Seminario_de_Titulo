\chapter{Introducción}
\label{cha:introduccion}


UX es .............
CX es .............

TX es una disciplina que corresponde a un caso particular de CX (A su vez relacionada con UX), 
pero con la particularidad de que estudia a los sentimientos, sensaciones, experiencias y muchas mas dimensiones 
asociadas a la relacion Turista - Viaje en los momentos antes, durante y despues del mismo, relacion 
que puede verse a su vez afectada por multiples factores, otorgandole una dimensionalidad infinita, 
considerando que en el ámbito de los viajes el turista puede verse afectado positiva o negativamente por 
tantas variables como factores influyen en nuestra vida diaria, teniendo ideas preconcebidas de los lugares que 
visitará incluso sin saberlo o haberse dado cuenta (sobre todo cuando se trata de publicidad o campañas que tienen
por objetivo que sin darte cuenta implanten una idea sobre determinado tema en las personas).\\

TX posee muchas definiciones distintas y muchos autores han intentado llegar a un concenso de que es
realmente, en que consiste, que factores estudia, de donde se origina, cuantas dimensiones contempla, 
entre muchos mas factores, pero lo que si podemos decir es que podríamos considerarlo un constructo 
Holístico y multidimensional, derivado como un caso particular de CX (concepto asociado directamente a UX) donde 
el Customer utiliza servicios como productos o sistemas asociados a viajes. TX Según Matus et al. (2023) se cree que,
 “la Experiencia Turística (TX) puede referirse a las percepciones subjetivas cognitivas, afectivas, sensoriales y 
 conativas, ya sean negativas o positivas, e influenciadas por factores situacionales, que tiene un turista al 
 interactuar con las marcas antes, durante y después del viaje, incluyendo sus resultados.”(p. 10). por lo que podemos 
