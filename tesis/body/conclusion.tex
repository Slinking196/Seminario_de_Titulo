\chapter{Conclusión}
\label{cha:conclusion}

Consideramos que este proyecto puede llegar a ser un aporte significativo para la industria del turismo, ya que al mejorar la experiencia del usuario en plataformas digitales, puede influir positivamente en la experiencia del turista (TX), lo cuál para ser un país que posee tantos sectores tan atractivos, es fundamental. Por esta misma razón, el análisis de un gran volumen de reseñas, y mediante el uso de técnicas de procesamiento de lenguaje natural y modelos de datos, nos permitirá identificar patrones y tendencias en las experiencias de los turistas, para luego poder implementar mejoras a través de ideas innovadoras y creativas, que puedan ser aplicadas en las plataformas digitales de turismo.

La experiencia del usuario (UX) y la experiencia del consumidor (CX) siempre han sido un desafío, ya que la satisfacción de un usuario o consumidor depende de muchos factores y cada persona es diferente, pero con el apoyo de un dataset de gran tamaño (500.000 reseñas) y el uso de técnicas de análisis avanzadas, podemos obtener una visión más clara y objetiva de las necesidades y preferencias de los turistas, lo que nos permitirá diseñar soluciones más efectivas y personalizadas.

Como trabajo futuro, se abren varias líneas de investigación prometedoras. Sería valioso aplicar esta metodología a datos de otras plataformas para realizar un análisis comparativo del sector, o incluso integrar datos en tiempo real para monitorear la experiencia del turista de forma dinámica. Además, se podría explorar el desarrollo de un sistema de recomendaciones personalizado que, basándose en el perfil y las preferencias extraídas del análisis, sugiera destinos y servicios que se alineen mejor con la experiencia que cada usuario busca. En definitiva, este proyecto subraya que para innovar en el sector turístico ya no basta con ofrecer un buen servicio, sino que es fundamental diseñar y gestionar cada punto de contacto digital con un profundo entendimiento de las necesidades y emociones del turista moderno.

