\chapter{Preprocesamiento de datos}
\label{cha:Preprocesamiento de datos}

En esta sección se describirá el proceso de preprocesamiento de datos llevado a cabo para preparar el conjunto de datos de reseñas de usuarios para su posterior análisis y trabajo con NLU, las técnicas que consideramos pertinentes, y las herramientas utilizadas para lograr una preparación de datos de calidad y de esta forma asegurar una base sólida de trabajo.\\ 

Como se recomienda de entre distintas fuentes bibliográficas como \cite{Agrawal2024TextPreprocessing}, el preprocesamiento de datos es un paso importante que conlleva distintas técnicas de preparación de datos para la luego utilización de los modelos NLU. A continuación se describirán las técnicas y herramientas utilizadas en esta fase de preprocesamiento del Dataset.

\section{\textbf{Reconocimiento de lenguaje}}
Utilizando librerías de Python como \textbf{langdetect} y realizando una previa división del dataset en distintos dataframes/archivos para trabajarlos de manera descentralizada, se procedió con el reconocimiento del lenguaje de los distintos comentarios encontrados ......................................... CONTINUAR EXPLICACION

\section{\textbf{Limpieza y separación de dataset}}

Luego se realizó un trabajo de selección de las columnas ded interés que aportarían verdadera información a las herramientas de NLU, de manera que trabajemos con un fataframe reducido y mas atomizado, pero sin perder información útil, asegurando una mejora de rendimiento de los modelos comparado con un trabajo de todas las columnas del dataset.

\section{\textbf{Lowercasing}}

Al momento de tener que trabajar directamente con la preparación de los \textbf{comentarios}, debimos  
\section{\textbf{Tokenización}} NO
\section{\textbf{Lematización}} NO
\section{\textbf{Eliminación de Stopwords}} NO
\section{\textbf{Normalización}}
\section{\textbf{Vectorización}}
\section{\textbf{ETC.........}}

