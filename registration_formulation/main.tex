\documentclass[12pt, letterpaper]{report} % Usa la clase report para documentos divididos en capítulos

% Paquete que da el formato al documento
\usepackage{../pucv_inf_2024}
\usepackage{lipsum}
\usepackage{indentfirst}
\usepackage{verbatim}
\usepackage{pdfpages}
\usepackage[table]{xcolor}

\addbibresource{../references.bib}
\usepackage{hyperref}

% Configurar biblatex para que no haga salto de página
\defbibheading{bibliography}[\refname]{\section*{#1}}

\begin{document}
    \pagenumbering{gobble}
    
    %% Portada para informe de Inscripción del Proyecto
    \begin{center}
        \textbf{PONTIFICIA UNIVERSIDAD CATÓLICA DE VALPARAÍSO}\\
        \textbf{FACULTAD DE INGENIERÍA}\\
        \textbf{ESCUELA DE INGENIERÍA INFORMÁTICA}

        \vspace{1cm}
        \textbf{ANÁLISIS DE LA EXPERIENCIA DEL TURISTA}\\
        \vspace{1cm}
        Fabrizzio Andrés Mura Lavarello, Matías Hernán Bugueño Bugueño
    \end{center}

    %% Introducción al Proyecto
    El uso de la tecnología en los últimos años ha aumentado considerablemente en la vida cotidiana de las personas,
    esto es debido a diversos factores como la globalización, el avance de la tecnología y la creciente necesidad de las 
    personas de acceder a información de manera rápida y eficiente. Según un estudio de McKinsey and Company, la pandemia 
    del COVID-19 aceleró la digitalización de las interacciones con clientes y operaciones internas en un promedio de tres a 
    cuatro años, y el desarrollo de productos digitales avanzó siete años en tan solo unos meses\cite{mckinsey2020}.\\

    Por esta misma razón se ha visto un aumento considerable del uso de plataformas digitales para la búsqueda de hospedaje, 
    viajes, sitios túristicos para visitar y otros servicios relacionados con el turismo una vez fue finalizada la pandemia,
    esto provoca que las plataformas de turismo online tengan que mejorar la experiencia de los usuarios contantemente para poder
    mantenerse competitivos en un mercado tan volátil y cambiante.

    %% Descripción del Proyecto
    \section*{Descripción del Proyecto}
    
    Actualmente las personas utilizan plataformas digitales como Booking\cite{booking2025}, 
    TripAdvisor \cite{tripadvisor2025}, Trivago\cite{trivago2025}, entre otras, para buscar y reservar servicios 
    de alojamiento, transporte, atracciones, vuelos y muchos otros servicios enfocados a todo lo que buscan 
    los turistas en sus viajes, ya sea por comodidad, entretención, seguridad, apuro o simplemente ocio.\\

    Es por esto que en este proyecto buscaremos realizar analisis de reseñas de estas plataformas, 
    obteniendo información con técnicas como webScrapping para recolectar datos de las reseñas de usuarios de 
    manera masiva de páginas como las anteriormente mencionadas, para luego aplicar técnicas para etiquetado 
    de datos con modelos de inteligencia artificial, y así poder identificar patrones, sentimientos, necesidades 
    o en general un analisis de datos que nos entregue información relevante, además de técnicas de procesamiento de lenguaje natural (NLP), todo con el objetivo final de poder sugerir mejoras en la 
    experiencia de usuario (UX) enfocada al turista (TX) .\\

    Se ha planteado la idea de que, con esta información cualitativa, se podria recabar información para 
    mejorar o generar una nueva plataforma/aplicación que mejore la experiencia del turista (TX) de muchas formas y que tal vez esto
    pueda tener repercusiones en la industria del turismo, preferente y esperablemente de manera positiva, teniendo además en cuenta que en esta industria se mueven muchas personas y dinero 
    constantemente y que sobre todo buscan la mejor experiencia en estos servicios y en caso de no cumplirse las expectativas simplemente se pierde 
    el cliente por mucho tiempo o para siempre, por lo que es un mercado muy volatil y donde hay que ser cautelosos.

    %% Objetivos del Proyecto
    \section*{Objetivos del proyecto:}
    
    A modo general, mejorar las recomendaciones de viajes, hospedaje y servicios relacionados al turismo, gracias a un 
    analisis exhaustivo de las reseñas de usuarios en plataformas como las mencionadas en el punto 
    \textbf{Descripción del proyecto} utilizando técnicas de procesamiento de lenguaje natural.

    Entrando mas a detalle en este proyecto, buscamos recolectar reseñas de usuarios de plataformas 
    turísticas digitales tales como Booking, TripAdvisor y/o Trivago utilizando técnicas de web 
    scraping de forma automatizada y masiva; preprocesar y limpiar los datos recolectados, aplicando 
    técnicas de normalización y filtrado para garantizar su calidad y utilidad en el análisis; 
    aplicar técnicas de procesamiento de lenguaje natural (NLP) para analizar el contenido textual 
    de las reseñas y extraer patrones lingüísticos relevantes; realizar el etiquetado y posterior 
    clasificación de las reseñas utilizando modelos de inteligencia artificial; realizar análisis de datos,
    identificando puntos críticos y oportunidades de mejora en la experiencia del turista (TX); y, 
    finalmente, proponer recomendaciones o mejoras para un posible diseño o rediseño de 
    plataformas/aplicaciones que mejoren la experiencia del turista.


    %% Cronograma Tentativo de Trabajo
    \section*{Plan de trabajo tentativo}

    El Proyecto se desarrollará en un periodo estimado de 4 meses, distribuyendo las actividades principales en las siguientes
    etapas:

    \begin{enumerate}
        \item \textbf{Investigación del Estado del Arte:} En esta etapa se llevará a cabo una revisión exhaustiva de los estudios
               previos y el análisis que han realizados diversos investigadores y empresas con respecto a la experiencia del usuario 
               en las plataoformas digitales de turismo, así como las técnicas de procesamiento de lenguaje natural y análisis de sentimientos 
               aplicadas en este contexto. Esta etapa tendrá una duración estimada de 2 semanas.

        \item \textbf{Recolección de Datos:} En esta etapa se llevará a cabo la recolección de reseñas de usuarios de plataformas turísticas 
               digitales utilizando técnicas de web scraping. Esta etapa tendrá una duración estimada de 2 semanas.

        \item \textbf{Preprocesamiento de Datos:} En esta etapa se realizará la limpieza de los datos recolectados, se llevará a cabo la normalización,
              la tokenización y la eliminación del ruido en los datos, esto para luego poder aplicar diversas técnicas de procesamiento de lenguaje natural.
              Esta etapa tendrá una duración estimada de 1.5 semanas.

        \item \textbf{Etiquetado Automático de Datos:} En esta etapa se aplicarán distintos modelos para el procesamiento de datos, como modelos de inteligencia
                artificial no supervisados para la clasificación de sentimientos y tópicos relacionados directamente a la experiencia del usuario. Esta etapa tendrá
                una duración estimada de 2 semanas.

        \item \textbf{Análisis de Resultados:} En esta etapa se realizará el análisis de los datos limpios y etiquetados, identificando patrones claves con apoyo de 
                modelos de inteligencia artificial supervisados para la predicción de las opiniones de los usuarios a la hora de ofrecer un servicio, como también
                modelos estadísticos para la identificación de tendencias y correlaciones en los datos. Esta etapa tendrá una duración estimada de 1.5 semanas.

        \item \textbf{Propuestas y Conclusiones:} En esta etapa se redactarán las recomendaciones basadas en los hallazgos de los análsis realizados, como también
                se diseñará un prototipo tentativo de mejoras o un nuevo software que mejore la experiencia del usuario en las plataformas digitales de turismo. 
                Esta etapa tendrá una duración estimada de 5 días.
                
        \item \textbf{Redacción del Informe Final:} En esta etapa se integrarán los resultados obtenidos, las conclusiones y las futuras líneas de desarrollo del proyecto,
                redactando el informe final del proyecto. Esta etapa tendrá una duración estimada de 2 semanas.
    \end{enumerate}

    %% Bibliografía
    \printbibliography[title={Referencias}, heading=bibliography]

\end{document}


