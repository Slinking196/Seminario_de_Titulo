\chapter{Introducción}
\label{cha:introduccion}

En el último siglo las tecnologias han avanzado a pasos agigantados, las nuevas tecnologias, dispositivos, el uso de IoT y Edge Computing, la IA generativa últimamente que ha generado cambios completos en la sociedad y que da para un survey completo de estado del arte, Paradigmas o arquitecturas como Zero Trust entre muchas otras tecnologias han cambiado la forma en que la sociedad vive, convive, se expresa, siente o incluso viaja. Es aquí donde debemos hablar sobre como las personas (a resumidas cuentas) perciben y sienten los touchpoints de las aplicaciones y/o sistemas, que aunque podria tener una definición mas amplia es donde podemos reconocer el concepto de UX.\\

En este trabajo de investigación realizaremos análisis de datos de turistas de la página de TripAdvisor\cite{tripadvisor-website}, realizaremos limpieza de datos, análisis de lenguaje natural de un dataset de quinientos mil comentarios de distintos usuarios-turistas que dejaron sus reseñas escritas luego de hospedarse en sitios en el contexto de viajes y turismo, para finalmente poder poder analizar la Experiencia de los Turistas (UX) y de esta forma destacar puntos positivos, negativos y oportunidades de mejora y así finalmente proponer soluciones posibles a dichos problemas sin que solapen los puntos positivos detectados y que nos permitan (o en el mejor de los casos se complementen) aprovechar las oportunidades de mejora detectadas, además de servir como procesamiento y analisis de datos para futuras o paralelas investigaciones sobre UX que requieran de estos datos o resultados de los mismos\\  

Utilizaremos, para realizar el Analisis Lenguaje Natural (NLP) y para un posterior Analisis de Datos (DA), el modelo de 7 constructos de MTX de Kim et al \cite{kim2012development}, donde se definen los constructos: 
\begin{enumerate}
    \item \textit{Hedonismo}
    \item \textit{Renovación (Recuperación o Descanso)}
    \item \textit{Cultura local}
    \item \textit{Significatividad}
    \item \textit{Conocimiento}
    \item \textit{Involucramiento}
    \item \textit{Novedad}
\end{enumerate}

Constructos que nos ayudarán a analizar y categorizar los comentarios de los turistas y así poder extraer conclusiones y propuestas de mejora, utilizando un sistema de analisis de lnguaje natural (comentarios) y clustering que determine el nivel [1 - 10] como escala likert de cada una de las 7 etiquetas que posee cada comentario, así de esta forma obteniendo la información completa (debido al caracter multidimensional del modelo) de los comentarios de las personas y poder asociar de mejor manera factores que tal vez no tendríamos en cuenta si no consideraramos el panorama completo, así asegurando una clasificación con un enfoque mas \textbf{Holístico} como se menciona en \cite{su151712765} y que finalmente derive en el reconocimiento de caracteristicas positivas, negativas y oportunidades de mejora, que a su vez permitirán plantear propuestas más efectivas y precisas para mejorar la experiencia del turista, todo gracias a un análisis de datos bien estructurado, metódico y exhaustivo.\\

