\chapter{Resumen}
\label{cha:resumen}

Este trabajo busca comprender, desde una perspectiva cercana y multidimensional, cómo los turistas experimentan y perciben el uso de plataformas digitales como Trivago durante la planificación y el desarrollo de sus viajes. Más allá de los aspectos técnicos, nos interesa identificar los factores emocionales, cognitivos, sensoriales y prácticos que inciden en la satisfacción y en la calidad de la experiencia turística.

Para ello, se analiza un conjunto de datos de reseñas y opiniones de usuarios, previamente recolectados por el equipo académico, aplicando técnicas de procesamiento de lenguaje natural (NLP) y modelos de análisis de datos. El objetivo es clasificar y agrupar las experiencias reportadas, detectando patrones, tendencias y problemáticas recurrentes en la interacción digital de los turistas.

A partir de los hallazgos obtenidos, se presentan recomendaciones orientadas a mejorar la usabilidad, accesibilidad y satisfacción en plataformas turísticas digitales. Se enfatiza la importancia de abordar la experiencia del turista (TX) desde una mirada integral, considerando tanto los aspectos emocionales y contextuales como los funcionales, con el fin de contribuir al diseño de herramientas más efectivas y humanas para el sector turismo. Finalmente, se plantean líneas futuras para la mejora continua de estas plataformas, basadas en la evidencia y en la experiencia real de los usuarios.

\section{Abstract}
\label{cha:abstract}

This work aims to understand, from a close and multidimensional perspective, how tourists experience and perceive the use of digital platforms like Trivago during the planning and execution of their trips. Beyond technical aspects, we are interested in identifying the emotional, cognitive, sensory, and practical factors that influence satisfaction and the quality of the tourist experience.

To achieve this, we analyze a dataset of user reviews and opinions, previously collected by the academic team, applying natural language processing (NLP) techniques and data analysis models. The objective is to classify and group the reported experiences, detecting patterns, trends, and recurring issues in the digital interaction of tourists.

Based on the findings, we present recommendations aimed at improving usability, accessibility, and satisfaction on digital tourist platforms. We emphasize the importance of addressing the tourist experience (TX) from a comprehensive perspective, considering both emotional and contextual aspects as well as functional ones, in order to contribute to the design of more effective and human-centered tools for the tourism sector. Finally, we propose future lines for the continuous improvement of these platforms, based on evidence and real user experiences.

% La presente investigación aborda el análisis y la mejora de la experiencia del turista (TX) en plataformas digitales, integrando conceptos de experiencia de usuario (UX) y experiencia del consumidor (CX) en el contexto turístico. Se parte de la premisa de que la satisfacción del turista está influenciada por múltiples dimensiones —cognitiva, afectiva, sensorial, conativa, intelectual y hedónica— y que la interacción con aplicaciones web especializadas constituye un factor clave en la percepción global del viaje.

% El estudio utiliza un conjunto de datos de reseñas y opiniones de usuarios recolectados previamente en plataformas como Trivago, aplicando técnicas de procesamiento de lenguaje natural (NLP) y análisis de datos. Se emplean modelos supervisados y no supervisados para clasificar, agrupar y analizar las experiencias reportadas, permitiendo identificar patrones, tendencias y problemáticas recurrentes en la interacción digital de los turistas.

% A partir de los resultados obtenidos, se proponen recomendaciones orientadas a optimizar la usabilidad, accesibilidad y satisfacción en plataformas turísticas, contribuyendo a un enfoque más holístico y multidimensional de la TX. El trabajo destaca la importancia de considerar factores emocionales, funcionales y contextuales en el diseño y evaluación de herramientas digitales para el sector turismo, y sugiere líneas futuras para la mejora continua basada en evidencia empírica y análisis automatizado.