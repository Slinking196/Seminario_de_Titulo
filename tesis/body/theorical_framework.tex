\chapter{Marco Teórico}
\label{cha:marco_teorico}

En esta sección, se explorarán los conceptos y teorías relevantes que sustentan la investigación. Se abordarán temas como la experiencia del usuario (UX), la experiencia del consumidor (CX) y la experiencia del turista (TX), así como sus interrelaciones y dimensiones.

\section{Usuario}
\label{sec:usuario}

El usuario es el individuo que interactúa con un sistema, producto o servicio. En el contexto de esta investigación, el usuario es el turista que utiliza plataformas digitales para planificar y llevar a cabo su viaje. Comprender al usuario implica analizar sus necesidades, expectativas y comportamientos a lo largo de su experiencia turística.

\section{Consumidor}
\label{sec:consumidor}

El consumidor es el individuo que adquiere y utiliza bienes o servicios. En el contexto de esta investigación, el consumidor es el turista que utiliza plataformas digitales para planificar y llevar a cabo su viaje. Comprender al consumidor implica analizar sus motivaciones, preferencias y comportamientos de compra a lo largo de su experiencia turística.

\section{Turismo}
\label{sec:turismo}

El turismo es una actividad que implica el desplazamiento de personas a lugares distintos de su entorno habitual, con fines de ocio, negocios u otros motivos. En el contexto de esta investigación, el turismo se refiere a la experiencia del viajero al utilizar plataformas digitales para planificar y llevar a cabo su viaje. Comprender el turismo implica analizar las dinámicas y tendencias que afectan la industria, así como las expectativas y comportamientos de los turistas en el entorno digital.

\section{Turista}
\label{sec:turista}

El turista es el individuo que viaja a un lugar distinto de su entorno habitual, con fines de ocio, negocios u otros motivos. En el contexto de esta investigación, el turista es el usuario que utiliza plataformas digitales para planificar y llevar a cabo su viaje. Comprender al turista implica analizar sus motivaciones, expectativas y comportamientos a lo largo de su experiencia turística.

\section{Experiencia del Usuario (UX)}
\label{sec:ux}

La experiencia del usuario (UX) se refiere a la percepción y respuesta de un usuario ante la interacción con un sistema, producto o servicio. Este concepto abarca aspectos como la usabilidad, accesibilidad y satisfacción del usuario. La UX se centra en el diseño y la mejora de los puntos de contacto (touchpoints) entre el usuario y el sistema, buscando optimizar la experiencia general del usuario.

\section{Experiencia del Consumidor (CX)}
\label{sec:cx}

La experiencia del consumidor (CX) es un concepto más amplio que incluye todas las interacciones que un cliente tiene con una marca a lo largo de su relación. Esto abarca no solo la interacción con productos y servicios, sino también la comunicación y el soporte al cliente. La CX se considera una extensión de la UX, ya que ambas se centran en la satisfacción y percepción del usuario.

CX, puede ser considerado por muchos autores como Laming \& Mason como una extensión de UX, altamente multidisciplinaria, que considera las respuestas de los usuarios tanto antes como durante y después de la interacción con una marca/compañía y durante toda una jornada [\cite{laming2014customer_experience}]. También mencionan cómo la CX debe ser definida como las experiencias tanto físicas como emocionales generadas antes, durante o después, pero producto de la interacción con un producto o servicio de una marca. Además, aparecen conceptos como el objetivo de las marcas para mejorar la CX o que el indicador clave es la satisfacción del cliente (Customer Satisfaction).

\section{Experiencia del Turista (TX)}
\label{sec:tx}

La experiencia del turista (TX) es un caso particular de la CX, que se centra en las interacciones y percepciones de los turistas antes, durante y después de su viaje. La TX abarca múltiples dimensiones, incluyendo emociones, sensaciones y experiencias relacionadas con el viaje. Este concepto es especialmente relevante en el contexto de la industria del turismo, donde la satisfacción del cliente puede verse afectada por diversos factores.
Según Matus et al. (2023), la TX puede definirse como "las percepciones subjetivas cognitivas, afectivas, sensoriales y conativas, ya sean negativas o positivas, e influenciadas por factores situacionales, que tiene un turista al interactuar con las marcas antes, durante y después del viaje, incluyendo sus resultados" (p. 10).

Sus dimensiones son:
\begin{itemize}
    \item Cognitiva: Se refiere a las percepciones y pensamientos del turista sobre su experiencia.
    \item Afectiva / Emocional: Incluye las emociones y sentimientos que surgen durante el viaje.
    \item Sensorial: Relacionada con las experiencias sensoriales, como vistas, sonidos y olores.
    \item Conativa / Intencional: Se refiere a las intenciones y comportamientos del turista antes, durante y después del viaje.
    \item Intelectual: Relacionada con el conocimiento y la comprensión del turista sobre su experiencia.
    \item Hedónica: Se refiere al placer y disfrute que el turista obtiene de su experiencia.
\end{itemize}