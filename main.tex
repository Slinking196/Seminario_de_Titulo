\documentclass[12pt, letterpaper]{report} % Usa la clase report para documentos divididos en capítulos

%Paquete que da el formato al documento
\usepackage{pucv_inf_2024}
\usepackage{lipsum}
\usepackage{indentfirst}
\usepackage{verbatim}
\usepackage{pdfpages}
\usepackage[table]{xcolor}

% Archivo de Referencias

% Glosario
\loadglsentries{glosario}
\makeglossaries
\addbibresource{referencias.bib}
\usepackage{hyperref}

% Parte el Documento
\begin{document}
%% TEX CON PORTADA DEL INFORME
% Portada para informes en asignaturas
\include{common/portada}

% Portada para informes de TFT a excepción del final de Proyecto
%\title{Título del Tema del TFT}
\author{Nombre Estudiante 1\\[10pt]Nombre Estudiante 2}
\newcommand{\prof}{Nombre Profesor}
\newcommand{\corprof}{Nombre Profesor Correferente}
\date{Julio 2024}


\begin{titlepage}
  \makeatletter
  \thispagestyle{empty} % Usa el estilo 'plain' definido para la página de título
    %\begin{center}
    
    \begin{picture}(0,0)
         \put(-75,-80){\includegraphics[width=20cm]{assets/encabezado.png}}
    \end{picture}
    %\end{center}

  
  \begin{flushright} % Alinea el contenido a la derecha
  \vspace*{\fill} % Centra verticalmente el contenido
    {\textcolor{titlecolor}{\LARGE\bfseries\MakeUppercase{
    \\[3cm]
    \@title}}}
  
    {\textcolor{titlecolor}{\Large\bfseries\textsc{\@author}}} \\[4cm]
    
    {\textcolor{titlecolor}{\large\bfseries\textsc{Profesor Guía: \prof}}} \\[10pt]
    {\textcolor{titlecolor}{\large\bfseries\textsc{Profesor Correferente: \corprof}}} \\[2cm]

    %\vspace*{\fill} % Centra verticalmente el contenido
    {\textcolor{titlecolor}{\bfseries\textsc{Informe de Avance/Final Seminario/Proyecto de Título\\[0.2cm]Nombre Carrera\\[0.2cm]\@date
    }}}

  \end{flushright}  
\end{titlepage}

% Portada informe final de proyecto 
%
\title{Título del Tema del TFT}
\author{Nombre Estudiante 1\\[10pt]Nombre Estudiante 2}
\renewcommand{\prof}{Nombre Profesor}
\renewcommand{\corprof}{Nombre Profesor Correferente}
\date{Julio 2024}


\begin{titlepage}
  \makeatletter
  \thispagestyle{empty} % Usa el estilo 'plain' definido para la página de título
    %\begin{center}
    
    \begin{picture}(0,0)
         \put(-75,-80){\includegraphics[width=20cm]{Portadas/imagenes/encabezado.png}}
    \end{picture}
    %\end{center}

  
  \begin{flushright} % Alinea el contenido a la derecha
  \vspace*{\fill} % Centra verticalmente el contenido
    {\textcolor{titlecolor}{\LARGE\bfseries\MakeUppercase{
    \\[3cm]
    \@title}}}
  
    {\textcolor{titlecolor}{\Large\bfseries\textsc{\@author}}} \\[3.5cm]
    
    {\textcolor{titlecolor}{\large\bfseries\textsc{Profesor Guía: \prof}}} \\[10pt]
    {\textcolor{titlecolor}{\large\bfseries\textsc{Profesor Correferente: \corprof}}} \\[2cm]

    %\vspace*{\fill} % Centra verticalmente el contenido
    {\textcolor{titlecolor}{\bfseries\textsc{Informe final de Proyecto de Título\\[0.2cm]Para Optar al Título Profesional de\\[0.2cm]Nombre del Título Profesional\\[0.2cm]\@date
    }}}

  \end{flushright}  
\end{titlepage}

% Portada para informe de Tesis de Posgrado
%\include{Portadas/portada_Tesis.tex}

%% RESUMEN/ABSTRACT DEL INFORME
\chapter{Resumen}
\label{cha:resumen}

Este trabajo busca comprender, desde una perspectiva cercana y multidimensional, cómo los turistas experimentan y perciben el uso de plataformas digitales como Trivago durante la planificación y el desarrollo de sus viajes. Más allá de los aspectos técnicos, nos interesa identificar los factores emocionales, cognitivos, sensoriales y prácticos que inciden en la satisfacción y en la calidad de la experiencia turística.

Para ello, se analiza un conjunto de datos de reseñas y opiniones de usuarios, previamente recolectados por el equipo académico, aplicando técnicas de procesamiento de lenguaje natural (NLP) y modelos de análisis de datos. El objetivo es clasificar y agrupar las experiencias reportadas, detectando patrones, tendencias y problemáticas recurrentes en la interacción digital de los turistas.

A partir de los hallazgos obtenidos, se presentan recomendaciones orientadas a mejorar la usabilidad, accesibilidad y satisfacción en plataformas turísticas digitales. Se enfatiza la importancia de abordar la experiencia del turista (TX) desde una mirada integral, considerando tanto los aspectos emocionales y contextuales como los funcionales, con el fin de contribuir al diseño de herramientas más efectivas y humanas para el sector turismo. Finalmente, se plantean líneas futuras para la mejora continua de estas plataformas, basadas en la evidencia y en la experiencia real de los usuarios.

\section{Abstract}
\label{cha:abstract}

This work aims to understand, from a close and multidimensional perspective, how tourists experience and perceive the use of digital platforms like Trivago during the planning and execution of their trips. Beyond technical aspects, we are interested in identifying the emotional, cognitive, sensory, and practical factors that influence satisfaction and the quality of the tourist experience.

To achieve this, we analyze a dataset of user reviews and opinions, previously collected by the academic team, applying natural language processing (NLP) techniques and data analysis models. The objective is to classify and group the reported experiences, detecting patterns, trends, and recurring issues in the digital interaction of tourists.

Based on the findings, we present recommendations aimed at improving usability, accessibility, and satisfaction on digital tourist platforms. We emphasize the importance of addressing the tourist experience (TX) from a comprehensive perspective, considering both emotional and contextual aspects as well as functional ones, in order to contribute to the design of more effective and human-centered tools for the tourism sector. Finally, we propose future lines for the continuous improvement of these platforms, based on evidence and real user experiences.

% La presente investigación aborda el análisis y la mejora de la experiencia del turista (TX) en plataformas digitales, integrando conceptos de experiencia de usuario (UX) y experiencia del consumidor (CX) en el contexto turístico. Se parte de la premisa de que la satisfacción del turista está influenciada por múltiples dimensiones —cognitiva, afectiva, sensorial, conativa, intelectual y hedónica— y que la interacción con aplicaciones web especializadas constituye un factor clave en la percepción global del viaje.

% El estudio utiliza un conjunto de datos de reseñas y opiniones de usuarios recolectados previamente en plataformas como Trivago, aplicando técnicas de procesamiento de lenguaje natural (NLP) y análisis de datos. Se emplean modelos supervisados y no supervisados para clasificar, agrupar y analizar las experiencias reportadas, permitiendo identificar patrones, tendencias y problemáticas recurrentes en la interacción digital de los turistas.

% A partir de los resultados obtenidos, se proponen recomendaciones orientadas a optimizar la usabilidad, accesibilidad y satisfacción en plataformas turísticas, contribuyendo a un enfoque más holístico y multidimensional de la TX. El trabajo destaca la importancia de considerar factores emocionales, funcionales y contextuales en el diseño y evaluación de herramientas digitales para el sector turismo, y sugiere líneas futuras para la mejora continua basada en evidencia empírica y análisis automatizado.

%% INDICE, LISTAS de FIGURAS/TABLAS/ALGORITMOS, GLOSARIO, ETC.
%% COMENTAR EN CASO DE NO QUERER INCLUIR ALGUNA DE ESTAS LISTAS/INDICES
\newpage\addcontentsline{toc}{chapter}{Índice General\vspace{-0.5cm}}
\tableofcontents % Este comando genera el índice

\newpage\addcontentsline{toc}{chapter}{Lista de Figuras\vspace{-0.5cm}}
\listoffigures % Índice de figuras
\listoftables
\newpage\addcontentsline{toc}{chapter}{Lista de Tablas\vspace{-0.5cm}}
%\listoftables  % Índice de tablas

% Imprimir Glosario
\newpage\addcontentsline{toc}{chapter}{Glosario}
\printglossary[type=\acronymtype]
\newpage
\pagenumbering{arabic} % Cambia a numeración arábiga para el resto del documento

% CONTENIDO PRINCIPAL DEL DOCUMENTO
\chapter{Introducción}
\label{cha:introduccion}

En el último siglo las tecnologias han avanzado a pasos agigantados, las nuevas tecnologias, dispositivos, el uso de IoT y Edge Computing, la IA generativa últimamente que ha generado cambios completos en la sociedad y que da para un survey completo de estado del arte, Paradigmas o arquitecturas como Zero Trust entre muchas otras tecnologias han cambiado la forma en que la sociedad vive, convive, se expresa, siente o incluso viaja. Es aquí donde debemos hablar sobre como las personas (a resumidas cuentas) perciben y sienten los touchpoints de las aplicaciones y/o sistemas, que aunque podria tener una definición mas amplia es donde podemos reconocer el concepto de UX.\\

En este trabajo de investigación realizaremos análisis de datos de turistas de la página de TripAdvisor\cite{tripadvisor-website}, realizaremos limpieza de datos, análisis de lenguaje natural de un dataset de quinientos mil comentarios de distintos usuarios-turistas que dejaron sus reseñas escritas luego de hospedarse en sitios en el contexto de viajes y turismo, para finalmente poder poder analizar la Experiencia de los Turistas (UX) y de esta forma destacar puntos positivos, negativos y oportunidades de mejora y así finalmente proponer soluciones posibles a dichos problemas sin que solapen los puntos positivos detectados y que nos permitan (o en el mejor de los casos se complementen) aprovechar las oportunidades de mejora detectadas, además de servir como procesamiento y analisis de datos para futuras o paralelas investigaciones sobre UX que requieran de estos datos o resultados de los mismos\\  

Utilizaremos, para realizar el Analisis Lenguaje Natural (NLP) y para un posterior Analisis de Datos (DA), el modelo de 7 constructos de MTX de Kim et al \cite{kim2012development}, donde se definen los constructos: 
\begin{enumerate}
    \item \textit{Hedonismo}
    \item \textit{Renovación (Recuperación o Descanso)}
    \item \textit{Cultura local}
    \item \textit{Significatividad}
    \item \textit{Conocimiento}
    \item \textit{Involucramiento}
    \item \textit{Novedad}
\end{enumerate}

Constructos que nos ayudarán a analizar y categorizar los comentarios de los turistas y así poder extraer conclusiones y propuestas de mejora, utilizando un sistema de analisis de lnguaje natural (comentarios) y clustering que determine el nivel [1 - 10] como escala likert de cada una de las 7 etiquetas que posee cada comentario, así de esta forma obteniendo la información completa (debido al caracter multidimensional del modelo) de los comentarios de las personas y poder asociar de mejor manera factores que tal vez no tendríamos en cuenta si no consideraramos el panorama completo, así asegurando una clasificación con un enfoque mas \textbf{Holístico} como se menciona en \cite{su151712765} y que finalmente derive en el reconocimiento de caracteristicas positivas, negativas y oportunidades de mejora, que a su vez permitirán plantear propuestas más efectivas y precisas para mejorar la experiencia del turista, todo gracias a un análisis de datos bien estructurado, metódico y exhaustivo.\\



\end{document}